\documentclass[10pt,conference]{ieeeconf}

\usepackage{graphicx}
\usepackage{algorithm}% http://ctan.org/pkg/algorithm
\usepackage{algpseudocode}% http://ctan.org/pkg/algorithmicx
\usepackage{xcolor}
\usepackage{subfigure}
\usepackage{amsmath}
\usepackage{pdfpages}
\usepackage{epstopdf}
\usepackage{color}
\usepackage{amssymb}
\usepackage{fixmath}
\usepackage{optidef}
\usepackage{subfig}
\usepackage{amssymb}
\usepackage{cite}
\usepackage{array,tabularx}
\usepackage{booktabs}
\usepackage{cleveref}

\usepackage{color}

\graphicspath{{./Images/}}

\providecommand{\rm}{\mathrm}

\IEEEoverridecommandlockouts
\begin{document}

\title{Literature Review for ``Multichannel Deconvolution of Skin Conductance Data: Concurrent Separation of Tonic and Phasic Component''}
% should be concise and informative to broad readers and attract attention from potential readers.


\author{Yuchen Jin, Jin Lu, Md. Rafiul Amin, and Rose T. Faghih \thanks{Yuchen Jin, Jin Lu, Md. Rafiul Amin, and Rose T. Faghih are with the Department of Electrical and Computer Engineering at the University of Houston, Houston, TX 77004 USA (e-mail: \tt\small  yjin4@uh.edu, jlu28@uh.edu, mamin@central.uh.edu, rtfaghih@uh.edu).} 
}

\maketitle

\begin{abstract}

The abstract should be left blank until we finish the main part of the paper.

\end{abstract}

\section{Introduction} \label{introduction}

Analyzing the electrical activities of neurons is (an important issue in the physiological field) {\color{red}of great interest in biomedical signal processing}. By (applying) {\color{red}placing} the electrodes, intramuscular needles or some other wearable devices, the neuron activities could be detected by different kinds of indirect methods including the electroencephalogram, electrodermal activities~\cite{savazzi2019estimation,jain2016compressed,amin2019robust}, electromyography signals~\cite{biagetti2016homomorphic}, calcium imaging, and some other approaches. The study of the neural activities could be utilized in many applications, like predicting people's emotions, monitoring the health-related abnormal events, and developing the brain-computer interface.

In this work, we concentrate on the study of electrodermal activities (EDA). EDA could be collected by measuring the skin conductance (SC) reacted to activities of people's sweat glands. The sensors for detecting EDA could be placed on either the hand palms, fingers, legs or feet. Previous works~\cite{fowles1981publication,amin2019robust} show that the EDA of different body parts is controlled by the universal neural signal from the autonomic nervous system (ANS). Within this hypothesis, the EDA could be modeled as a group of multichannel SC signals, where one channel represents the collected SC from a specific body part. For any channel $n$, the signal could be formulated by
\begin{align}
  y_{\rm{SC}_n}(t) = p_n(t) + s_n(t) + \nu_n(t),
\end{align}
%
{\color{red} $y_{\rm{SC}_n}(t)$  n should be italicized}

where $\mathbf{s}_n(t)$ is the tonic component representing the influence of the body thermoregulation {\color{red}and also the general arousal of the body} , $p_n(t)$ is the phasic component thought as reflected from the activity of ANS, and $v_n(t)$ is the additive noise. Different from the slowly drifting tonic component, the phasic component is fast-varying and periodic {\color{red}(delete 'periodic',because it is not accurate to say phasic signal is periodic)}, which makes it possible to separate the two signals by building two different mathematical models. Both the tonic component and the phasic component could be regarded as a linear composition of different basis, or a group of kernels convoluted on spike signals in time series. The interpretation for SC could be formulated as a deconvolution problem which requires a solution for the spike signals based on knowing the composed signal $y_{\rm{SC}_n}(t)$. {\color{red}for the last sentence: It is possible to formulate a deconvolution problem with the physiological interpretation of SC, which requires a solution for the ANS activation given the composed signal  $y_{\rm{SC}_n}(t)$  is known.  }

\subsection{Deconvolution}

The neural stimuli from the ANS is though as spike signals. While after the signal is conducted to the sweat glands, the signals are smoothly filtered due to the (physiological effects) {\color{red} physiological systems}. After applying the deconvolution, the estimated neural stimuli could be used for denoising, emotion predictions, or health monitoring. \textit{Caruelle et al.}~\cite{caruelle2019use} summarizes the applications of the EDA analysis in recent years. For more details, \textit{Posada et al.}~\cite{posada2020innovations} provides a review on the previous deconvolution methods with the EDA data. Similar to EDA, the other kinds of signals could be also used for studying the neuron activities by deconvolution. In \cref{tab:literature}, we inspect on several works on deconvolution with different data. For the simplest case, the spike signal could be retrieved by applying a threshold and performing the local maximum searching~\cite{kaur2016remote,subramanian2019systematic}. In most works, the deconvolution is solved by formulating an optimization problem based on a linear model, including the Bateman function~\cite{savazzi2019estimation,greco2014electrodermal,greco2015cvxeda,amin2019tonic,hernando2017feature,wickramasuriya2019skin}, the AR-1 model~\cite{friedrich2017fast}, and the state-space model~\cite{kazemipour2017fast,amin2019robust}. 

The complexity of the deconvolution differs in (different works) {\color{red}The complexity of the deconvolution differs in different applications/problems instead of works?} . In \cite{hernando2017feature}, the linear model is composed of several fixed basis. In \cite{greco2014electrodermal,greco2015cvxeda,amin2019tonic,wickramasuriya2019skin,kazemipour2017fast,amin2019robust}, {\color{red}The work in [11] does not solve for the SCR shape parameters. Please check all the literatures in the sentence and provide the accurate information}although there is a single basis, the shape of the basis could be learned. Particularly, some researches assume that the linear model is totally unknown. By only giving an initial guess, the modeling function is also optimized by blind deconvolution~\cite{kaur2016remote,friedrich2017fast}. (There is no best option for modeling the forward function.){\color{red} Rafiul said is it based on any literature?} For a large scale application, the problem requires to be formulated properly according to the computational consumption.

Most of the mentioned works are dealing with single channel data, while in \cite{friedrich2017fast,amin2019robust}, a multichannel scheme is introduced for improving the robustness of the predictions. In the practice, observations are mixed with the noise, which may influence the accuracy of the deconvolved spikes. The motivation of introducing multichannel data is based on the hypothesis that the same neural stimuli is shared by the data acquired by different observations. When the multichannel data is deconvolved jointly, the noise from different channels could be attenuated by each other, thus the problem could converge on a better solution.

\begin{table*}[htbp]
  \centering
  \normalsize
  \caption{A summary of the previous works about deconvolution.}
  \label{tab:literature}
  \begin{tabular}{m{0.14\textwidth}<{\raggedright}m{0.14\textwidth}<{\raggedright}m{0.65\textwidth}}
    \toprule
    \multicolumn{1}{c}{Study} & \multicolumn{1}{c}{Data} & \multicolumn{1}{c}{Methodology} \\ \midrule
    \textit{Friedrich et. al.}~\cite{friedrich2017fast} & Calcium imaging data & The conversion from the calcium spikes to the observed fluorescenc signals is described as an AR-p model. The authors propose an algorithm, OASIS for solving the constrained optimization problem in the AR-1 case. \\
    \textit{Kazemipour et. al.}~\cite{kazemipour2017fast} & Calcium imaging data & Develop a compressible state-space model for describing the data. The spikes are designed as the innovation sequence of the state. The optimization problem is solved by a nested Expectation-Maximization algorithm. \\
    \textit{Kaur et. al.}~\cite{kaur2016remote} & Electro-cardiogram & Introduce two methods. The first method is based on curve fitting and local maximum searching. The second method is Blind Deconvolution, where the kernel and the spikes are optimized alternatively. \\
    \textit{Mesin}~\cite{mesin2019single} & Electromyogram & Model the EMG signal by motor unit kernels. The deconvolution is performed by solving an $\ell_1$ regularized optimization problem by the iterative reweighted least square algorithm. \\
    \textit{Antink et. al.}~\cite{friedrich2017fast} & ECG, ABP and PPG & The author use the blind deconvolution to solve the heart beat spikes. This paper introduces a multi-observation scheme by assuming a universal spike data. \\
    \textit{Subramanian et. al.}~\cite{subramanian2019systematic} & EDA & Use a low-pass FIR filter to isolate the tonic component, the spike is found by searching local maximum in phasic component. \\ 
    \textit{Savazzi et. al.}~\cite{savazzi2019estimation} & EDA & The tonic component and noise is removed by a low-pass FIR filter. The phasic component is modeled by Bateman function. The spike is acquired by optimization. \\ 
    \textit{Jain et. al.}~\cite{jain2016compressed} & EDA & Involve the compress sensing into the deconvolution. The optimization is performed on the differential signal to eliminate the influence of sudden changes of the tonic component. \\
    \textit{Greco et. al.}~\cite{greco2014electrodermal} & EDA & The tonic component is estimated by Gaussian smoothing and interpolation. The spike is solved by modeling the phasic component with Bateman function. \\ 
    \textit{Greco et. al.}~\cite{greco2015cvxeda}, \textit{Amin and Faghih}~\cite{amin2019tonic} & EDA & Model the phasic component and the tonic component by Bateman function and spline function respectively, then formulate the deconvolution as a joint constrained convex optimization problem. \\ 
    \textit{Hernando et. al.}~\cite{hernando2017feature} & EDA & The tonic and phasic components are decomposed by two groups of pre-defined bases. The phasic component basis is designed by using Bateman function. \\
    \textit{Wickramasuriya et. al.}~\cite{wickramasuriya2019skin} & EDA & The tonic component is isolated by using cvxEDA~\cite{greco2015cvxeda}. The phasic component is decomposed by a basis based on Bateman function. The deconvolution is solved by a hybrid algorithm based on GCV and FOCUSS+. \\
    \textit{Amin and Faghih}~\cite{amin2019robust} & EDA & The authors propose a multichannel deconvotion scheme. The tonic component is removed by using cvxEDA~\cite{greco2015cvxeda} on each channel, and the phasic component is formulated by a state-space model. The deconvolution is solved by the hybrid algorithm GCV-FOCUSS+. \\ \bottomrule
  \end{tabular}
\end{table*}

\subsection{Tonic separation}
 
Previously there some works on the tonic component separation. In \cite{van1967skin}, the raw EDA signal is decomposed as absolute change in conductance, change as a ratio of the basic conductance level (BLC), and log conductance change. In \cite{green2014development}, the author proposes a method of automated analysis of SCR data in the contexts of event-related cognitive tasks and nonspecific responding to complex stimuli. \cite{greco2014electrodermal} formulates the separation as a quadratic programming problem, where the phasic component and the tonic component are decomposed by the basis of the biexponential Bateman functions and the B-spline functions respectively. To solve the problem that the tonic component may overlap each other, \textit{Vartak et. al.} {\color{red}(missing the reference)}proposes a mathematical model fitting procedure is used to separate these overlapping components. Features are extracted using the mathematical model fitting procedure. In~\cite{amin2019tonic}, the author proposes a generalized-cross-validation-based block coordinate descent approach. In this approach, the tonic component is regarded as a summation of a series of P shifted and weighted cubic B-spline functions. Then the problem is solved by minimizing an $\ell_p$ penalized loss function. Especially, inspired by compress sensing, \textit{Jain et. al.}~\cite{jain2016compressed} builds a model for separating the difference of the signal instead of the raw signal. This idea is effective for eliminating the interference caused by sudden changes of the tonic component.

In some previous works, the tonic component separation and the phasic component deconvolution are performed individually~\cite{greco2014electrodermal,amin2019robust}. Since both the modeling method and the inversion method in \cite{amin2019robust} are designed for the multichannel case, the previous tonic separation works designed for single-channel data may cause deviation for the solution. {\color{red} (In addition to the motivation, we are interested in multi-channel deconvolution inorder to obtain more accurate results in presence of noise. As tonic component separation is performed separately, inaccuracy in separation in presence of artifact and noise may lead to inaccurate deconvolution of ANS activation. A multi-channel tonic and phasic decomposition may lead to a more robust decomposition in presence of noise.)} In our work, we extend the multichannel deconvolution into the joint optimization for solving the tonic component in the multichannel case. With this scheme, the neural stimuli from the ANS is though to be shared, while the solution for the composition of tonic basis is assumed to be different in different channels.

\bibliographystyle{ieeetr}
\bibliography{ref}

\end{document}