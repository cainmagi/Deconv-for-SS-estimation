\documentclass[10pt,conference]{ieeeconf}

\usepackage{graphicx}
\usepackage{algorithm}% http://ctan.org/pkg/algorithm
\usepackage{algpseudocode}% http://ctan.org/pkg/algorithmicx
\usepackage{xcolor}
\usepackage{subfigure}
\usepackage{amsmath}
\usepackage{pdfpages}
\usepackage{epstopdf}
\usepackage{color}
\usepackage{amssymb}
\usepackage{fixmath}
\usepackage{optidef}
\usepackage{subfig}
\usepackage{amssymb}

\graphicspath{{./Images/}}

\providecommand{\rm}{\mathrm}

\IEEEoverridecommandlockouts
\begin{document}

\title{Literature Review for ``Multichannel Deconvolution of Skin Conductance Data: Concurrent Separation of Tonic and Phasic Component''}
% should be concise and informative to broad readers and attract attention from potential readers.


\author{Yuchen Jin, Jin Lu, Md. Rafiul Amin, and Rose T. Faghih \thanks{Yuchen Jin, Jin Lu, Md. Rafiul Amin, and Rose T. Faghih are with the Department of Electrical and Computer Engineering at the University of Houston, Houston, TX 77004 USA (e-mail: \tt\small  yjin4@uh.edu, jlu28@uh.edu, mamin@central.uh.edu, rtfaghih@uh.edu).} 
}

\maketitle

\begin{abstract}

The sudden occurrence of the health-related abnormal events has been a critical issue with respect to the report of worldwide deaths. A reliable and robust health monitoring method is significant. Currently the electrical activities (EDA) have been thought as effective for detecting abnormal health patterns. This report summarizes the works related to analyzing the skin conductance (SC) data. The data is viewed as a composition of the tonic signal and the multichannel phasic signal, where the tonic component is smooth and the phasic signal is periodic. After separating the phasic signal from the raw data, it is deconvoluted by solving an inverse problem of a state-space model. Both the component separation and the deconvolution are formulated by regularized optimization problems. We will focus on the jointly multichannel inversion for the component separation and the deconvolution.

\end{abstract}

\section{Introduction} \label{introduction}

The multichannel SC signal could be viewed as in the following form:
\begin{align}
  \mathbf{y}_{\rm{SC}_n}(t) = \mathbf{p}_n(t) + \mathbf{s}_n(t) + \mathbf{v}_n(t),
\end{align}

where $\mathbf{s}_n(t)$ is the tonic component representing the influence of the body thermoregulation, $\mathbf{p}_n(t)$ is the phasic component thought as reflected from the activity of the autonomic nervous system (ANS), and $\mathbf{v}_n(t)$ is the additive noise. $n$ is the channel number. Different from the slow-varying tonic component, the phasic component is fast-varying and periodic, which makes it possible to separate the two signals. A multichannel state-space control system could be built for describing how neural stimuli (neural spiking activity) get converted into the phasic component. The inverse problem is designed based on the forward system.

\section{Methods}

The main workflow of the interpretation for SC could be divided into three phases:

\begin{enumerate}
  \item Separating the tonic component and the phasic component from the original raw data.
  \item Deconvolve the phasic component to recover the neural stimuli.
  \item Interpret the neural stimuli, like estimating the human stress from the data.
\end{enumerate}

We would only concentrate on the first and the second step. The target of our work is designing a concurrent deconvolution and tonic component separation method for the multichannel SC data.

\subsection{Component separation}

Previously there some works on the tonic component separation. \cite{greco2014electrodermal} formulates the separation as a quadratic programming problem, where the phasic component and the tonic component are decomposed by the basis of the biexponential Bateman functions and the B-spline functions respectively. In~\cite{amin2019tonic}, the author proposes a generalized-cross-validation-based block coordinate descent approach. In this approach, the phasic component is modeled by a second-order equation, while the tonic component is regarded as a summation of series of P shifted and weighted cubic B-spline functions. Then the problem is solved by minimizing an $\ell_p$ penalized loss function.

\subsection{Deconvolution}

The deconvolution could be formulated as an optimization problem. \cite{wickramasuriya2019skin} focuses on the single channel deconvolution problem, which is penalized and constrained. The author uses a coordinate descent approach to solve the problem and maintain the constraint alternatively. In \cite{hernando2017feature}, the author solve the signal separation and the deconvolution jointly. The forward modeling system is approximated by pre-defined biexponential waveforms. \cite{amin2019robust} proposes a state-space control system for formulating the multichannel forward problem, and combines the GCV and FOCUSS+ algorithms together. Unlike \cite{amin2019tonic}, the proposed algorithm does not require to set fixed penalizing parameters, but incorporating the problem of finding the best penalization as a part of the optimization.

\subsection{Conclusion and further extension}

In previous works, the tonic component separation and the phasic component deconvolution are performed individually. Since both the modeling method and the inversion method in \cite{amin2019robust} are designed for the multichannel case, the previous tonic separation works designed for single channel data may be not accurate. The basic idea is replacing the forward modeling method in \cite{amin2019tonic} by the multichannel version in \cite{amin2019robust}. Another inspection is that in \cite{amin2019robust}, the rise and decay times ($\tau_r$, $\tau_d$) are fixed. If time is avaliable, we may explore the possibility to make the $\tau_r$ and $\tau_d$, i.e. the waveforms parameters learnable.

\bibliographystyle{ieeetr}
\bibliography{ref}

\end{document}