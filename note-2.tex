\documentclass[]{article}

\usepackage{amsmath,amssymb,amsbsy}
\usepackage[letterpaper,total={6in,8in}]{geometry}
%opening
\title{Review notes on tonic signal separation}
\author{Jin Lu}

\begin{document}

\maketitle

\begin{enumerate}
  \item Decomposing skin conductance into tonic and phasic components~\cite{lim1997decomposing}
  \begin{itemize}
    \item A method of resolving this problem using a modelling technique that takes advantage of the stereotyped nature of the within-subject SCR waveform. A four-parameter sigmoid-exponential SCR model that describes the entire response, was developed and extended to five-, six- and eight-parameter skin conductance SC models.
  \end{itemize}
  \item Estimation of arousal using decomposed skin conductance features - biomed 2009~\cite{vartak2009estimation}
  \begin{itemize}
    \item This separation of components is more challenging when the responses overlap each other as they do when responding within shorter inter-stimulus interval. A mathematical model fitting procedure is used to separate these overlapping components. Features are extracted using the mathematical model fitting procedure. These features are further used, to classify the EDR signal into high versus low arousal responses.
  \end{itemize}
  \item Skin conductance as a measure of tonic and phasic arousal~\cite{van1967skin}
  \begin{itemize}
    \item Phasic arousal seems to be measurable by skin conductance changes. In experimental practice data on skin conductance changes (EDR) are reported in a variety of scales: absolute change in conductance, change as a ratio of the basic conductance level (BLC), and log conductance change.
  \end{itemize}
  \item Development and validation of an unsupervised scoring system (Autonomate) for skin conductance response analysis~\cite{green2014development}
  \begin{itemize}
    \item A method of automated analysis of SCR data in the contexts of event-related cognitive tasks and nonspecific responding to complex stimuli. The primary goal of the method is to accurately measure the classical trough-to-peak amplitude of SCR in a fashion closely matching manual scoring. 
  \end{itemize}
  \item A Compressed Sensing Based Decomposition of Electrodermal Activity Signals~\cite{jain2016compressed}
  \begin{itemize}
    \item It shows how simple preprocessing followed by a novel compressed sensing based decomposition can mitigate the effects of the undesired noise components and help reveal the underlying physiological signal. 
  \end{itemize}
\end{enumerate}

\bibliographystyle{ieeetr}
\bibliography{ref}

\end{document}
