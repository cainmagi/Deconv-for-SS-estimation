\documentclass[]{article}

\usepackage{amsmath,amssymb,amsbsy}
\usepackage{color}
%opening
\title{Multichannel Deconvolution of Skin Conductance Data: Concurrent Speration of Tonic and Phasic Component}
\author{Jin Lu}

\begin{document}

\maketitle

The method for tonic and phasic decomposition is  a generalized-cross-validation-based block coordinate descent approach. Firstly, we need to formulate the model of phasic and tonic component. We use a second order differential equation model for the relationship between the phasic component and the neural stimuli (neural spiking activity). Thus, we can get the solution to the differential equation. It is the sum of an initial state decreasing term and the convolution result of the stimuli and the system response. Then, we model the tonic component as a summation of series of P shifted and weighted cubic B-spline functions. We also need to model the discrete sampling process for the SC signal. It can be seen as the discrete time SC function with an additive white Gaussian noise. Finally, we can represent the sampled data vector y with the neural stimuli by the following equation
\begin{align}
  \mathbf{y} = \mathbf{A}_\tau y_{p_0} + \mathbf{B}_\tau \mathbf{u} + \mathbf{C} \mathbf{q} + \boldsymbol{\nu},
\end{align}

After we filter the high frequency noise and downsample the filtered signal, we formulate the following optimization problem.
\begin{subequations}
  \renewcommand{\theequation}
  {\theparentequation-\arabic{equation}}
  \begin{align}
  \min\limits_{\mathbf{u}, \tau, \mathbf{q}}~&\left\lVert \mathbf{y} - \mathbf{A}_\tau y_{p_0} - \mathbf{B}_\tau \mathbf{u} - \mathbf{C} \mathbf{q} - \boldsymbol{\nu} \right\lVert^2_2 + \lambda_1 \lVert \mathbf{q} \rVert_2^2, \\
  \mathrm{s.t.}~& \tau^{\mathrm{min}} \leqslant \tau \leqslant \tau^{\mathrm{max}}, \\
  & \mathbf{u} \succcurlyeq \mathbf{0},~ \lVert \mathbf{u} \rVert_0 \leqslant N, \\
  & \mathbf{C}\mathbf{q} \preccurlyeq \mathbf{y}.
  \end{align}
\end{subequations}

Then, we include the l2-norm penalization term with regularization parameter to avoid over-fitting. Then, we get the following equation
\begin{subequations}
  \renewcommand{\theequation}
  {\theparentequation-\arabic{equation}}
  \begin{align}
  \min\limits_{\mathbf{u}, \tau, \mathbf{q}}~&\left\lVert \mathbf{y} - \mathbf{A}_\tau y_{p_0} - \mathbf{B}_\tau \mathbf{u} - \mathbf{C} \mathbf{q} - \boldsymbol{\nu} \right\lVert^2_2 + \lambda_1 \lVert \mathbf{q} \rVert_2^2 + \lambda_2 \lVert \mathbf{u} \rVert_p^p, \\
  \mathrm{s.t.}~& \tau^{\mathrm{min}} \leqslant \tau \leqslant \tau^{\mathrm{max}}, \\
  & \mathbf{u} \succcurlyeq \mathbf{0},~ \lVert \mathbf{u} \rVert_0 \leqslant N, \\
  & \mathbf{C}\mathbf{q} \preccurlyeq \mathbf{y}.
  \end{align}
\end{subequations}

{\color{red} haha} We use block coordinate descent algorithm to estimate all the parameters. Here, we omitted the procedure of the algorithm. After the implementation of the algorithm for several random values of system parameters, we can get the estimated result.

\end{document}
